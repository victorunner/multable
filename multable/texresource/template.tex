%\documentclass[border=2mm]{standalone}
\documentclass[class=scrreprt, fontsize=9pt, pagesize=off, paper=a4]{standalone}

\usepackage{pythontex}

\usepackage{tikz}
\usetikzlibrary{matrix}
\usetikzlibrary{positioning}
\usetikzlibrary{calc}

\usepackage{fontspec}
\usepackage{polyglossia}
\setdefaultlanguage{churchslavonic}
\setotherlanguages{english}
\usepackage{churchslavonic}
\newfontfamily\churchslavonicfont[Script=Cyrillic,Ligatures=TeX]{PonomarUnicode.otf}


\begin{document}

\input{importinput.tex}

\begin{tikzpicture}
[
square matrix/.style={
    matrix of nodes,
    column sep=-\pgflinewidth,
    row sep=-\pgflinewidth,
    nodes in empty cells,
    nodes={
        draw,
        rectangle,
        minimum size=8mm,
        anchor=center,
        align=center,
        inner sep=0pt
    },
    column 1/.style={nodes={fill=cyan!30}},
    row 1/.style={nodes={fill=cyan!30}},
  }
]

\begin{pycode}
def print_slavonic_number(number, include_arabic_number=True):
    s = rf'\cuNum{{{number}}}'

    if include_arabic_number:
      s += rf'\textsubscript{{{number}}}'

    print(s)

MAX_NUMBER = input_params['max_number']
SHOW_ARABIC = input_params['show_arabic']
LANGUAGE = input_params['language']


print(r'\matrix[square matrix] {')
for row in range(MAX_NUMBER + 1):
    for col in range(MAX_NUMBER + 1):
        if (row, col) != (0, 0):
            if row and col:
                val = row*col
            else:
                val = row + col

            if LANGUAGE == 'ARABIC':
                print(val)
            else:
                print_slavonic_number(val, SHOW_ARABIC)

        if col != MAX_NUMBER:
            print('&')
    print(r'\\')
print(r'};')
\end{pycode}
\end{tikzpicture}

\end{document}
